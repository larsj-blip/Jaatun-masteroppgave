\section{Sustainability analysis}
The motivation behind this thesis is rooted in sustainability. Enabling the usage of coding techniques that can reduce power consumption on existing hardware aligns well with sustainability goal number 9 of the UN sustainable development goals: 'Build resilient infrastructure, promote inclusive and sustainable industrialization and foster innovation'. 

The UN sustainability goals are a set of goals that the entirety of the UN have adopted. It is a common framework used by the member nations to make better choices with respect to the environment, both physical and social. The goals include gender and racial equality as well as reducing emissions from industry.

Just like approximate computing is a tradeoff between precision and processing speed/power consumption, implementing \emph{all} of the sustainability goals requires making some tradeoffs between the difference categories. 

All the categories represent unquestionable quality of life improvements for all humans, though enacting them may result in contraticting results. For instance, goal 1 is to end poverty in all its forms everywhere. So a factory gets established, which creates several jobs for the empoverished residents at this location. However, the factory produces runoff that contaminates a nearby water source, going against goal number 6 of ensuring availability and sustainability of water and sanitation for all. 

The following analysis has been created using a lightweight version of the SusAF framework~\cite{SusAF_website}



